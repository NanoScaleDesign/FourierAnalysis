\newpage
\section{Coursework}
\subsection{Task}
There are many interesting things about Fourier analysis, whether in terms of the mathematics or applications.
This course attempts to give you a solid grounding and understanding of Fourier analysis and then builds on these fundamentals in order to give you a taste of different topics related to its theory and applications.
Due to the wide range of topics available it is necessary to pick-and-choose to some extent about what topics are covered.
The aim of this coursework is to give you the opportunity to study a topic of your own choice.

Your task is therefore as follows:
Imagine that you are the teacher of this course.
Considering a topic about Fourier analysis of your choice, create a short document explaining to your students about Fourier analysis in the context of the topic you have chosen, and create sufficient challenges with worked solutions for your peers to test their understanding of the topic.
You may choose to cover a topic that has already been covered in the course, but teach the students about it using an interesting application as motivation. Or you may wish to teach students about a specific application of Fourier analysis, building upon fundamental knowledge they already have. Or you may choose to cover a new topic that has not been covered by the course so far. The choice is yours.

Points will be awarded on the basis of creativity, demonstration of knowledge, quality of explanation and accuracy.

\subsection{Submission}
Your submission should contain:
\begin{itemize}
    \item Either text (guideline: 2 to 4 pages) or video (guideline: 10 minutes)
    \item A list of references (either included in the text or submitted with the video)
    \item Challenges with their fully-worked solutions (ie, not just the final answer) (guideline: 2 challenges)
\end{itemize}

Submission is electronic, and may be in text or video format. For text-based formats, submission may be in any format, including PDF, LibreOffice, MS Word, Google docs, Latex, etc\ldots If you submit a PDF, please also submit the source-files used to generate the PDF. For video-based formats, a link to a video-sharing site (like YouTube) or a link to download the video file is required. Challenges and their solutions should be submitted in written form, even if a video is submitted.

Submit the materials by \textbf{email} to the teacher \textbf{before the class on 24 January 2018} with the subject ``[\coursenospace] Coursework''. I will confirm in the class that I received your coursework. If you cannot attend the class, you must request confirmation of receipt when you send the email.

Late submission:\\
By 17:00 on 25 January 2018: 90\% of the final mark.\\
By 17:00 on 31 January 2018: 50\% of the final mark.\\
Later submissions cannot be considered.
