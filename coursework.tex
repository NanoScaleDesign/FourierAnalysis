\newpage
\section{Coursework}
Fourier analysis is a large subject with a wide-range of applications. This coursework is designed to give you the opportunity to follow your personal interest and investigate in depth an area of Fourier analysis of your choice.

The task is as follows:

\textbf{1)} Create a document, explaining about any area of Fourier analysis that interests you. For example, you could consider how Fourier analysis is applied in the field you're pursuing for your degree, or you could explore some mathematics related to Fourier Analysis that you find particularly interesting. The document should be \textbf{at least 1 full page}, including any necessary figures, mathematics and references.

\textbf{2)} Create at least 1 challenge to accompany your report, so someone reading your document can test their knowledge.

\textbf{3)} Include \textbf{fully worked} solutions to challenges you make (ie, not only the final answer, but clearly show the steps involved in order to achieve the final answer).

Please choose a subject that does not directly repeat what is covered by the course up to the submission deadline. For example, we will be convering convolution, so if you choose to write about that subject, please make your content distinct from what is already covered by the course.

I may (or may not) choose to incorporate some aspects of the submissions into teaching of the final 1 or 2 classes.

\subsection{Submission}
You must submit \textbf{both a paper and electronic version}. Submit the materials by \textbf{email} to the teacher by \textbf{10:30 on 11 January 2017} with the subject ``[Fourier Analysis] Coursework'' and \textbf{bring a paper copy to the class on that day}.

The electronic version may be in any format, including LibreOffice, MS Word, Google docs, Latex, etc\ldots If you submit a PDF, please also submit the source-files used to generate the PDF.

Late submission:\\
By 10:00 on 12 January 2017: 90\% of the final mark.\\
By 10:00 on 16 January 2017: 50\% of the final mark.\\
Later submissions cannot be considered.

\subsection{Marking}
Marks will be assigned based on the degree to the report fulfills the following criteria:
\begin{itemize}
    \item Understanding: Clearly demonstrate your understanding of what you write about. You can do this by, for example, mathematically solving for a relevant case or explaining with words how it applies in different situations.
    \item Relevance: A subject that makes use of Fourier series or Fourier Transforms.
    \item Originality: It should be your own work. Also, you must \textbf{cite all references, as well as images and text taken from other sources}.
    \item Level: The subject should be pitched at a level whereby anyone else in the class could learn about the subject based on your report. Be sure to explain in reasonable depth.
    \item Accuracy: The explanation should be accurate and clear.
\end{itemize}
