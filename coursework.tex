\newpage
\section{Coursework}
Fourier analysis is a large subject with a wide-range of applications. This coursework is designed to give you the opportunity to follow your personal interest and investigate in depth an area of Fourier analysis of your choice.

The task is as follows:

\textbf{1)} Create a document, explaining about any area of Fourier analysis that interests you. For example, you could consider how Fourier analysis is applied in the field you're pursuing for your degree, or you could explore some mathematics related to Fourier Analysis that you find particularly interesting. The document should be \textbf{at least 2 pages}, including any necessary figures, mathematics and references.

\textbf{2)} Create at least 1 challenge per page to accompany your document, so someone reading your document can test their knowledge.

\textbf{3)} Include solutions to the challenges you make.

Before the submission date below, we will be covering convolution and the basics of the discrete Fourier Transform. Therefore, please choose a subject other than these (although you may study ahead and build upon these subjects if you wish). Of course, any subject you choose will have a basis in the mathematics of Fourier series and/or Fourier transforms, and thus a some overlap will be unavoidable.

I may (or may not) choose to incorporate some aspects of the submissions into teaching of the final 1 or 2 classes.

\subsection{Submission}
You must submit \textbf{both a paper and electronic version}. Submit the materials by \textbf{email} to the teacher by \textbf{10:30 on 11 January 2017} with the subject ``[Fourier Analysis] Coursework'' and \textbf{bring a paper copy to the class on that day}.

The electronic version may be in any format, including LibreOffice, MS Word, Google docs, Latex, etc\ldots If you submit a PDF, please also submit the source-files used to generate the PDF.

Late submission:\\
By 10:00 on 12 January 2017: 90\% of the final mark.\\
By 10:00 on 16 January 2017: 50\% of the final mark.\\
Later submissions cannot be considered.

\subsection{Marking}
Marks will be assigned based on
\begin{itemize}
    \item Relevance: A subject that makes use of Fourier series or Fourier Transforms.
    \item Originality: It should be your own work. You must \textbf{cite all references, as well as images and text taken from other sources}.
    \item Level: The subject should be pitched at a level whereby anyone else in the class could learn about the subject based on your work.
    \item Explanation: An accurate explanation described with depth and clarity.
\end{itemize}
