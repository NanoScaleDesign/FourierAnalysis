%%%%%%%%%%%%%%%%%%%%%%%%%%%%%%%%%
\newpage
%%%%%%%%%%%%%%%%%%%%%%%%%%%%%%%%%
\section{Partial derivatives}

\subsection*{Challenge}
Determine $u_t$ and $u_{xx}$ for the equation

\begin{equation}
    u(x,t) = 5tx^2 + 3t - x
\end{equation}

To check your answer, substitute $x=3$ and $t=2$ into your answers, as appropriate.

\subsection*{Solution}
$u_t$:\\
\solint{t}{4fb068}

$u_{xx}$:\\
\solint{x}{53502b}




%%%%%%%%%%%%%%%%%%%%%%%%%%%%%%%%%
\newpage
%%%%%%%%%%%%%%%%%%%%%%%%%%%%%%%%%
\section{Heat equation: Periodicity}

\subsection*{Resources}
\begin{itemize}
    \item Book: Section 1.13.1 (\url{https://see.stanford.edu/materials/lsoftaee261/book-fall-07.pdf})
    \item Lecture 4 from 37:00 onwards: (\url{https://www.youtube.com/watch?v=n5lBM7nn2eA}), continuing at the start of lecture 5 (\url{https://www.youtube.com/watch?v=X5qRpgfQld4})
\end{itemize}

\subsection*{Comment}
Here we can learn about an application of Fourier series to solve partial differential equations. This problem was one of the motivations for Fourier to develop the idea of Fourier series.

The motivation for the equation $u_t = \frac{1}{2} u_{xx}$ described in the notes is complicated somewhat by the interpretation in terms of equivalences between electrical and thermal capacitance. If this is not so clear then don't worry about it. At a minimum you should understand the following:
\begin{itemize}
    \item Heat flow is proportional to the gradient of the temperature.
    \item Heat accumulates within a unit volume when the rate of heat flow into that volume is greater than the rate of heat flow out of that volume.
\end{itemize}

\subsection*{Challenge}
The following statements concern a heated ring with circumference 1 and temperature distribution described by $u(x,t)$. Add the points of the statements that are defined by the system to be true:

1 point: $\displaystyle u(x,t) = u(x,t)$

2 points: $\displaystyle u(x,t) = u(x,t+1)$

4 points: $\displaystyle u(x,t) = u(x,t+2)$

8 points: $\displaystyle u(x,t) = u(x+1,t)$

16 points: $\displaystyle u(x,t) = u(x+1,t+1)$

32 points: $\displaystyle u(x,t) = u(x+1,t+2)$

64 points: $\displaystyle u(x,t) = u(x+2,t)$

128 points: $\displaystyle u(x,t) = u(x+2,t+1)$

256 points: $\displaystyle u(x,t) = u(x+2,t+2)$

512 points: The temperature distribution is periodic in space but not time

1024 points: The temperature distribution is periodic in time but not space

2048 points: The temperature distribution is periodic in both space and time

4096 points: The temperature distribution is periodic neither in space nor time


\subsection*{Solution}
\solint{y}{2259d1}




%%%%%%%%%%%%%%%%%%%%%%%%%%%%%%%%%
\newpage
%%%%%%%%%%%%%%%%%%%%%%%%%%%%%%%%%
\section{Heat equation on a ring: derivation}
\label{sec:heateqnfc}

\subsection*{Resources}
\begin{itemize}
    \item Video: \url{https://www.youtube.com/watch?v=yAOCibHPgLA}
    \item Book: Section 1.13.1 (\url{https://see.stanford.edu/materials/lsoftaee261/book-fall-07.pdf})
    \item Lecture 4 from 37:00 onwards: (\url{https://www.youtube.com/watch?v=n5lBM7nn2eA}), continuing at the start of lecture 5 (\url{https://www.youtube.com/watch?v=X5qRpgfQld4})
\end{itemize}

\subsection*{Challenge}
Starting from the heat (diffusion) equation $u_t = u_{xx}/2$, show that the general solution to the heat equation on a ring is given by
\begin{equation}
    u(x,t) = \sum_{n=-\infty}^{n=\infty} c_n(0) e^{-2 \pi^2 n^2 t} e^{i 2 \pi n x}
\end{equation}
and write an expression for $c_n(0)$ in terms of the initial temperature distribution $u(x,0)$.

\subsection*{Solution}
Please compare with your peers during discussion time and ask if there is anything you do not understand.




%%%%%%%%%%%%%%%%%%%%%%%%%%%%%%%%%
\newpage
%%%%%%%%%%%%%%%%%%%%%%%%%%%%%%%%%
\section{Heat equation on a ring: calculation}
\label{sec:heateqn}

\subsection*{Challenge}
Consider an initial heat distribution around a ring. Relative to ambient temperature, the initial temperature distribution follows a cosine distribution with the peak temperature at $x=0$.

1. Write an expression for the initial relative temperature distribution, $u(x,0)$, assuming the ring has a circumference of 1 unit.

2. Determine $c_{n=\pm 1}(0)$ and $c_{n \ne 1}(0)$

3. Write an expression for the temperature distribution as a function of position and time $u(x,t)$.

You can see an animation of the solution here:\\
\url{https://raw.githubusercontent.com/NanoScaleDesign/FourierAnalysis/master/Images/cosrelax.gif}

\subsection*{Solution}
1. You should find that your temperature distribution satisfies the result:\\
$u(0.2,0) = 0.309$

2.\\
$c_{n=\pm 1}(0)$\\
\soltwodp{b}{06ac46}

$c_{n \ne 1}(0)$\\
\soltwodp{a}{690969}

3. To check your answer you may substitute $x=0.3$ and $t=0.01$ into your solution:\\
$u(0.3,0.01) = -0.25$
