\documentclass[a4paper]{book} %{article}

\usepackage{fullpage} % Package to use full page
\usepackage{parskip} % Package to tweak paragraph skipping
\usepackage{tikz} % Package for drawing
\usepackage{amsmath}
\usepackage{hyperref}
\usepackage[numbered]{bookmark} % For numbering of challenges in bookmark pane of PDF viewer
\def\UrlBreaks{\do\/\do-}
\usepackage[absolute]{textpos}
\setlength{\TPHorizModule}{1mm}
\setlength{\TPVertModule}{1mm}
\usepackage{tikz}
\usepackage{siunitx}
\usepackage{bm} % Bold math \bm command
\usepackage{datetime} % Time
\usepackage[UKenglish]{isodate} % UK-format date
\usepackage{ctable} % Thick table lines
\usepackage{amssymb} % Realspace symbol


\newcommand{\six}[1]{\SI[parse-numbers=false]{X}{#1}}
\graphicspath{{Images/}}

% Discussion times
\newcommand{\disctime}{10:30 to 12:00 }
\newcommand{\discdays}{Wednesdays }
\newcommand{\discroom}{W4-529}
\newcommand{\discexam}{8th February 2017} % Final exam
\newcommand{\course}{Fourier Analysis }
\newcommand{\courseurl}{fourier-analysis}
\newcommand{\nensei}{3rd}
\newcommand{\timebox}{\vfill Study-time (from end of previous challenge to end of this challenge): \underline{\hspace{1cm} minutes}}
\newcommand{\assstart}{26 October}


\newcommand{\hash}[2]{MD5(#1\_X) = #2\ldots}
% Add timetable?

\title{Fourier Analysis\\Autumn 2016}
\author{James Cannon, Kyushu University}
\date{\today}

\begin{document}

\begin{titlepage}
    \begin{center}
        \vspace*{1cm}

        \Huge
        \textbf{Fourier Analysis}

        Autumn 2016

        \vspace{1.5cm}
        \Large
        Last updated:\\\today \ at \currenttime

        \vspace{4.0cm}
        \LARGE
        James Cannon\\Kyushu University
         \vfill

        \normalsize
        \url{http://www.jamescannon.net/teaching/\courseurl}
        \vspace{0.2cm}
        \small
        \url{http://raw.githubusercontent.com/NanoScaleDesign/FourierAnalysis/master/fourier_analysis.pdf}
        \vspace{0.5cm}

        License: \emph{CC BY-NC 4.0}.

    \end{center}
\end{titlepage}

\setcounter{chapter}{-1}

\tableofcontents

\chapter{Course information}
\newpage
\section{This course}
This is the Autumn 2016 \course course studied by \nensei-year undegraduate international students at Kyushu University.

\subsection{How this works}
\begin{itemize}
    \item In contrast to the traditional lecture-homework model, in this course the learning is self-directed and active via publicly-available resources.
    \item Learning is guided through solving a series of carefully-developed challenges contained in this book, coupled with suggested resources that can be used to solve the challenges with instant feedback about the correctness of your answer.
    \item There are no lectures. Instead, there is discussion time. Here, you are encouraged to discuss any issues with your peers, teacher and any teaching assistants. Furthermore, you are encouraged to help your peers who are having trouble understanding something that you have understood; by doing so you actually increase your own understanding too.
    \item Discussion-time is from \disctime on \discdays at room \discroom.
    \item Peer discussion is encouraged, however, if you have help to solve a challenge, always make sure you do understand the details yourself. You will need to be able to do this in an exam environment. If you need additional challenges to solidify your understanding, then ask the teacher. The questions on the exam will be similar in nature to the challenges. If you can do all of the challenges, you can get 100\% on the exam.
    \item Every challenge in the book typically contains a \textbf{Challenge} with suggested \textbf{Resources} which you are recommended to utilise in order to solve the challenge. A \textbf{Solution} is made available in encrypted form. If your encrypted solution matches the encrypted solution given, then you know you have the correct answer and can move on. For more information about encryption, see section \ref{sec:hashes}. Occasionally the teacher will provide extra \textbf{Comments} to help guide your thinking.
    \item For deep understanding, it is recommended to study the suggested resources beyond the minimum required to complete the challenge.
    \item The challenge document has many pages and is continuously being developed. Therefore it is advised to view the document on an electronic device rather than print it. The date on the front page denotes the version of the document. You will be notified by email when the document is updated.
    \item A target challenge will be set each week. This will set the pace of the course and defines the examinable material. It's ok if you can't quite reach the target challenge for a given week, but then you will be expected to make it up the next week.
    \item You may work ahead, even beyond the target challenge, if you so wish. This can build greater flexibility into your personal schedule, especially as you become busier towards the end of the semester.
    \item Your contributions to the course are strongly welcomed. If you come across resources that you found useful that were not listed by the teacher or points of friction that made solving a challenge difficult, please let the teacher know about it!
\end{itemize}

\subsection{Assessment}
In order to prove to outside parties that you have learned something from the course, we must perform summative assessments. This will be in the form of

\begin{equation*}
    \text{overall score} = (0.5 E_F + 0.3 E_M + 0.2 C) (0.9 + P/10)
\end{equation*}

Your final score is calculated as Max($E_F$, overall score), however you must pass the final exam ($\ge 60\%$) to pass the course.

\begin{itemize}
    \item $E_F$ = \% correct on final exam
    \item $E_M$ = \% correct on mid-term exam
    \item $C$ = \% grade on course-work
    \item $P$ = participation calculated as $P = (F/N_D)(A/N_D)(L/N_L)$ where each terms is as follows
        \begin{itemize}
            \item $F$ = Number of weeks where feedback form is submitted 24 hours before discussion time.
            \item $A$ = Number of discussion classes attended
            \item $N_D$ = Number of discussion classes held
            \item $L$ = Number of times that your collected challenge log is satisfactory. This means:
                \begin{itemize}
                    \item Available on request.
                    \item Your calculations are clearly shown.
                    \item It contains evidence of trying to keep up with the target challenge. Short-term fluctuations in completing challenges are fine (eg, if you had trouble understanding material to overcome some challenges this week) but the long-term trend should be more-or-less to keep up with the target challenge.
                \end{itemize}
            \item $N_L$ = Number of times that your challenge log is collected.
        \end{itemize}
\end{itemize}

Note that $P$ is only calculated from \assstart. If $N - F \le 2$ then $F$ is treated as being equal to $N$ (ie, you can forget twice). You can be counted as attending the class even if you are not present if the reason for not attending is unavoidable (eg, health reasons) and you inform the teacher in advance.

Please also note that, since late arrivals disrupt the class by preventing intended pairing of students, attendance of a discussion class will be only counted as partial if you are more than a minute or two late (eg, 9 minutes late out of a 90-minute discussion class will count as attending only 90\% of the class). Therefore, if you will be unavoidably late, you need to let the teacher know in advance. To allow for unexpected delays, for up to two late arrivals you will be considered to have attended 100\% of the discussion time.



\subsection{What you need to do}
\begin{itemize}
    \item Prepare a challenge-log in the form of a workbook or folder where you can clearly write the calculations you perform to solve each challenge. This will be a log of your progress during the course and will be occasionally reviewed by the teacher.
    \item You also need to submit feedback at \url{https://goo.gl/forms/Djl4FEZcJLMpipsY2} 24 hours before the discussion time starts. Here you can let the teacher know about any difficulties you are having and if you would like to discuss anything in particular.
    \item Please bring a wifi-capable internet device to class, as well as headphones if you need to access online components of the course during class. If you let me know in advance, I can lend computers and provide power extension cables for those who require them (limited number).
\end{itemize}

\newpage
\section{Timetable}

\begin{center}
    \begin{tabular}{|c|c|c|c|}
        \hline
        & \textbf{Discussion} & \textbf{Target} & \textbf{Note} \\ \specialrule{.1em}{.05em}{.05em}
        \textbf{1}  & 11 Oct & -            &                             \\ \hline
        \textbf{2}  & 18 Oct & 2.11         &                             \\ \hline
        \textbf{3}  & 25 Oct & 3.10         &                             \\ \hline
        \textbf{4}  & 27 Oct & 3.13         & Friday class in W4-766      \\ \specialrule{.1em}{.05em}{.05em}
        \textbf{5}  & 10 Nov & 3.18         & Friday class in W4-766      \\ \hline
        \textbf{6}  & 16 Nov & 3.27         & Thursday class at 16:40     \\ \hline
        \textbf{7}  & 22 Nov & 3.31         &                             \\ \hline
        \textbf{8}  & 29 Nov & 4.2          &                             \\ \specialrule{.1em}{.05em}{.05em}
        \textbf{9}  & 6 Dec  & Midterm exam &                             \\ \hline
        \textbf{10} & 13 Dec & 4.8          &                             \\ \hline % NT: Give students more of a rest after the mid-term
        \textbf{11} & 20 Dec & 4.11         &                             \\ \specialrule{.1em}{.05em}{.05em}
        \textbf{12} & 10 Jan & 4.14         & Coursework assignment       \\ \hline
        \textbf{13} & 17 Jan &              &                             \\ \hline % 5.7
        \textbf{14} & 24 Jan & Coursework   & Coursework submission       \\ \specialrule{.1em}{.05em}{.05em}
        \textbf{15} & 7 Feb  & Final exam   &                             \\ \hline
    \end{tabular}
\end{center}

\newpage
\section{Hash-generation}
\label{sec:hashes}

Some solutions to challenges are encrypted using MD5 hashes. In order to check your solution, you need to generate its MD5 hash and compare it to that provided. MD5 hashes can be generated at the following sites:

\begin{itemize}
    \item Wolfram alpha: (For example: md5 hash of ``q1.00'') \url{http://www.wolframalpha.com/input/?i=md5+hash+of+\%22q1.00\%22}
    \item \url{www.md5hashgenerator.com}
\end{itemize}

Since MD5 hashes are very sensitive to even single-digit variation, you must enter the solution \emph{exactly}. This means maintaining a sufficient level of accuracy when developing your solution, and then entering the solution according to the format suggested by the question. Some special input methods:

\begin{center}
\begin{tabular}{|l|l|}
    \hline
    \textbf{Solution} & \textbf{Input} \\ \hline
    $5 \times 10^{-476}$ & 5.00e-476 \\
    $5.0009 \times 10^{-476}$ & 5.00e-476 \\
    $-\infty$ & -infinity (never ``infinite'')\\
    $2 \pi$ & $6.28$ \\
    i & im(1) \\
    2i & im(2) \\
    1 + 2i & re(1)im(2) \\
    -0.0002548 i & im(-2.55e-4) \\
    1/i = i/-1 = -i & im(-1) \\
    $e^{i2\pi}$ [$= cos(2 \pi) + isin(2 \pi) = 1 + i0 = 1$] & 1.00 \\
    $e^{i\pi/3}$ [$= cos(\pi/3) + isin(\pi/3) = 0.5 + i 0.87$] & re(0.50)im(0.87) \\
    Choices in order A, B, C, D & abcd \\
    \hline
\end{tabular}
\end{center}

The first 6 digits of the MD5 sum should match the first 6 digits of the given solution.


\chapter{Periods and frequencies}
\section{Period at 50 THz}

\subsection*{Resources}
\begin{itemize}
    \item Video: \url{https://www.youtube.com/watch?v=v3CvAW8BDHI}
\end{itemize}

\subsection*{Challenge}
A signal is oscillating at a frequency of 50 THz.  What is the period?

\subsection*{Solution}
\solscitwodp{q}{3faf81}




%%%%%%%%%%%%%%%%%%%%%%%%%%%%%%%%%
\newpage
%%%%%%%%%%%%%%%%%%%%%%%%%%%%%%%%%

\section{Frequency with k=1}

\subsection*{Comment}
Note that we're working in radians here. From now on a factor of $2 \pi$ will be included in the oscillations so that $Sin(2 \pi t)$ will complete 1 cycle in 1 second.  (If you calculator defaults to degrees, be sure to change it to radians for this course.)

\subsection*{Challenge}
What is the frequency of $Sin(2 \pi k t)$, where $t$ is time in seconds and $k=1$?

\subsection*{Solution}
(Hz)

\soltwodp{e}{720149}




%%%%%%%%%%%%%%%%%%%%%%%%%%%%%%%%%
\newpage
%%%%%%%%%%%%%%%%%%%%%%%%%%%%%%%%%
\section{Frequency with k=2}

\subsection*{Challenge}
What is the frequency of $Sin(2 \pi k t)$, where $t$ is time in seconds and $k=2$?

\subsection*{Solution}
(Hz)

\soltwodp{r}{96ba66}




%%%%%%%%%%%%%%%%%%%%%%%%%%%%%%%%%
\newpage
%%%%%%%%%%%%%%%%%%%%%%%%%%%%%%%%%
\section{The meaning of $k$}

\subsection*{Challenge}
Considering the previous two challenges, what does $k$ physically represent in those challenges?

\subsection*{Solution}
Please compare your answer with your partner in class or discuss with the teacher.




%%%%%%%%%%%%%%%%%%%%%%%%%%%%%%%%%
\newpage
%%%%%%%%%%%%%%%%%%%%%%%%%%%%%%%%%
\section {Smallest period with k=1}

\subsection*{Resources}
\begin{itemize}
    \item Book: 1.3 (\url{https://see.stanford.edu/materials/lsoftaee261/book-fall-07.pdf})
    \item Video (14m10s to 17m00s): \url{https://youtu.be/1rqJl7Rs6ps?t=14m10s}
\end{itemize}

\subsection*{Challenge}
What is the smallest period of $sin(2 \pi k t)$, where $t$ is time in seconds and $k=1$?

\subsection*{Solution}
(s)

\soltwodp{w}{25c4fb}




%%%%%%%%%%%%%%%%%%%%%%%%%%%%%%%%%
\newpage
%%%%%%%%%%%%%%%%%%%%%%%%%%%%%%%%%

\section{Smallest period with k=2}

\subsection*{Resources}
\begin{itemize}
    \item Book: 1.3 (\url{https://see.stanford.edu/materials/lsoftaee261/book-fall-07.pdf})
    \item Video (14m10s to 17m00s): \url{https://youtu.be/1rqJl7Rs6ps?t=14m10s}
\end{itemize}

\subsection*{Challenge}
What is the smallest period of $sin(2 \pi k t)$, where $t$ is time in seconds and $k=2$?

\subsection*{Solution}
(s)

\soltwodp{t}{bb995f}




%%%%%%%%%%%%%%%%%%%%%%%%%%%%%%%%%
\newpage
%%%%%%%%%%%%%%%%%%%%%%%%%%%%%%%%%

\section{Phase}

\subsection*{Comments}
Another important concept is phase. For a simple sine signal $\theta(t) = sin(2 \pi t)$, at $t=0$ the angle $\theta$ is zero, but one can shift the phase (starting point) of the signal by effectively making the sine-curve non-zero at $t=0$. Another way to think about it is to say the sine curve doesn't reach zero until a time $t-\phi$ where $\phi$ is the phase-shift added.

\subsection*{Challenge}
Place the following four graphs in the following order:\\
$Sin(2 \pi t + \pi/2)$\\
$Sin(2 \pi t - \pi/2)$\\
$Sin(2 \pi t + \pi/4)$\\
$Sin(2 \pi t + 2 \pi)$\\

\includegraphics{phase_shift.png}

\subsection*{Solution}
\solstr{i}{5c0e8b}



%%%%%%%%%%%%%%%%%%%%%%%%%%%%%%%%%
\newpage
%%%%%%%%%%%%%%%%%%%%%%%%%%%%%%%%%
\section{Amplitude}

\subsection*{Comments}
Another important concept is amplitude. $Sin(2 \pi t)$ has an amplitude of 1, but this can be easily modified to go between $\pm A$ by multiplication with $A$.

\subsection*{Challenge}
The following 4 graphs correspond to the equation $A Sin(2 \pi k t)$ with variation in the values of $A$ and $k$. What is the sum of the values of $A$ for the following graphs? 

\includegraphics{amplitude.png}

\subsection*{Solution}
\solint{u}{6bce05}




%%%%%%%%%%%%%%%%%%%%%%%%%%%%%%%%%
\newpage
%%%%%%%%%%%%%%%%%%%%%%%%%%%%%%%%%
\section{Periodic and non-periodic signals}
\label{sec:periodic}

\subsection*{Resources}
\begin{itemize}
    \item Video: \url{https://www.youtube.com/watch?v=F_pdpbu8bgA}
    \item Book: 1.3 (\url{https://see.stanford.edu/materials/lsoftaee261/book-fall-07.pdf})
\end{itemize}

\subsection*{Challenge}
The list below contains periodic and non-periodic signals. Sum the points of the signals below that are \emph{periodic}.

1 point: $x(t) = t^2$\\
2 points: $x(t) = Sin(t)$\\
4 points: $x(t) = Sin(2 \pi t)$\\
8 points: $x(t) = Sin(2 \pi t + t)$\\
16 points: $x(t) = Sin(2 \pi t) + Sin(t)$\\
32 points: $x(t) = Sin(5 \pi t) + Sin(2 \pi t)$\\
64 points: $x(t) = Sin(5 \pi t) + Sin(37 \pi t)$\\
128 points: $x(t) = Sin(5 \pi t) + Sin(37.01 \pi t)$\\
256 points: $x(t) = Sin(5 \pi t) + Sin(\sqrt{2} \pi t)$\\
512 points: $x(t) = Sin(5 \sqrt{2} \pi t) + Sin(\sqrt{2} \pi t)$

\subsection*{Solution}
\solint{u}{5d906b}




%%%%%%%%%%%%%%%%%%%%%%%%%%%%%%%%%
\newpage
%%%%%%%%%%%%%%%%%%%%%%%%%%%%%%%%%
\section{Making non-periodic signals from periodic signals}

\subsection*{Challenge}
It is not immediately intuitive that it is possible to make a non-periodic signal by simply adding two periodic signals. Referring to the previous challenge, in no-more than 1 paragraph, explain how this is possible.

\subsection*{Solution}
Please compare your answer with your partner or ask the teacher in class.




%%%%%%%%%%%%%%%%%%%%%%%%%%%%%%%%%
\newpage
%%%%%%%%%%%%%%%%%%%%%%%%%%%%%%%%%
\section{Fundamental frequency}

\subsection*{Challenge}
Considering the periodic signals in challenge \ref{sec:periodic} in order of increasing point-score, calculate the fundamental frequency and period of the last periodic signal in the list.

\subsection*{Solution}
Frequency (Hz) (be careful about rounding up or down to 2 decimal places):\\
\soltwodp{a}{ac6698}

Period (s):\\
\soltwodp{b}{6fdff9}

\chapter{Fourier Series}
% NT: Specifically direct students to watch videos and read specific materials; especially the 2nd lecture video.
\section{Introduction to Fourier coefficients}
\label{sec:fouriercoeffintro}

\subsection*{Resources}
\begin{itemize}
    \item Book: 1.4 (\url{https://see.stanford.edu/materials/lsoftaee261/book-fall-07.pdf})
    \item Video: Lecture 2 (\url{https://www.youtube.com/watch?v=1rqJl7Rs6ps})
\end{itemize}

\subsection*{Challenge}
Deduce in a simple way the Fourier coefficients $a_1$ and $b_1$ in the Fourier series
\begin{equation}
    \sum_{k=1}^{N} a_k cos(2 \pi k t) + b_k sin(2 \pi k t)
\end{equation}

for a signal made up of multiple sine signals
\begin{equation}
    \sum_{k=1}^{N} A_k sin(2 \pi k t + \phi_k)
\end{equation}

for the following cases:
\begin{enumerate}
    \item $N=1$, $k=1$, $A_1=1$, $\phi_1=0$
    \item $N=1$, $k=1$, $A_1=1$, $\phi_1=\pi/2$
    \item $N=1$, $k=1$, $A_1=1$, $\phi_1=\pi/5$
\end{enumerate}

Hint: Using $sin(A+B) = sin(A) cos(B) + cos(A) sin(B)$ it should is possible to find the answer without resorting to complex calculation.

\subsection*{Solutions}
\begin{enumerate}
    \item MD5(o\_$a_k$) = a2c1fe\ldots, MD5(p\_$b_k$) = de80c6\ldots
    \item MD5(a\_$a_k$) = 718a6c\ldots, MD5(s\_$b_k$) = f86f0c\ldots
    \item MD5(d\_$a_k$) = 93d647\ldots, MD5(f\_$b_k$) = 9a7b58\ldots
\end{enumerate}

\timebox




%%%%%%%%%%%%%%%%%%%%%%%%%%%%%%%%%
\newpage
%%%%%%%%%%%%%%%%%%%%%%%%%%%%%%%%%

\section{Even and odd functions}
% NT: Need to add practise about even/odd functions occuring as part of fourier functions, maybe in terms of trig cases

\subsection*{Resources}
\begin{itemize}
    \item Wikipedia: \url{https://en.wikipedia.org/wiki/Even_and_odd_functions}
\end{itemize}

\subsection*{Challenge}
Referring in part to the cases in challenge \ref{sec:fouriercoeffintro}, sum the points of all the following TRUE statements:

1 point: Case 1 is an odd function

2 points: Case 1 is an even function

4 points: Case 2 is an odd function

8 points: Case 2 is an even function

16 point: Case 3 is an odd function

32 points: Case 3 is an even function

64 points: $f(x)=Sin(x)$ is an odd function

128 points: $f(x)=Sin(x)$ is an even function

256 points: $f(x)=Cos(x)$ is an odd function

512 points: $f(x)=Cos(x)$ is an even function

1024 points: $f(x)=x$ is an odd function

2048 points: $f(x)=x$ is an even function

\subsection*{Solution}
X

\hash{g}{6a18c0}

\timebox




%%%%%%%%%%%%%%%%%%%%%%%%%%%%%%%%%
\newpage
%%%%%%%%%%%%%%%%%%%%%%%%%%%%%%%%%

\section{Fourier coefficients of sin(x)}
\label{sec:fcsinx}

\subsection*{Resources}
\begin{itemize}
    \item Book: 1.4 (\url{https://see.stanford.edu/materials/lsoftaee261/book-fall-07.pdf})
    \item Video: Lecture 2 (\url{https://www.youtube.com/watch?v=1rqJl7Rs6ps})
\end{itemize}

\subsection*{Comments}
You should be able to follow the derivation of the formula for Fourier coefficients ($C_k$'s) in the video. Feel free to seek help if you have trouble.

\subsection*{Challenge}
1. Write $Sin(x)$ in exponential form.

2. Expand the Fourier series for the function $f(x)$ within the limit $k=-1$ to $k=1$. % NT: Check evaluation by making this a separate question and substituting to check answer.

3. Deduce the Fourier coefficients ($C_k$'s) for the function $Sin(x)$. \emph{This should be possible by inspection, rather than any significant calculation.} % NT: Include C_2, etc too

\subsection*{Solutions}
$C_{-1}$: \hash{h}{28f251}

$C_{0}$: \hash{j}{4fd3f6}

$C_{1}$: \hash{k}{e82a2a}

\timebox




%%%%%%%%%%%%%%%%%%%%%%%%%%%%%%%%%
\newpage
%%%%%%%%%%%%%%%%%%%%%%%%%%%%%%%%%

\section{Fourier Coefficients of 1 + sin(x)}
\label{sec:fcsinxp1}

\subsection*{Resources}
\begin{itemize}
    \item Book: 1.4 (\url{https://see.stanford.edu/materials/lsoftaee261/book-fall-07.pdf})
    \item Video: Lecture 2 (\url{https://www.youtube.com/watch?v=1rqJl7Rs6ps})
\end{itemize}

\subsection*{Comments}
You should be able to follow the derivation of the formula for Fourier coefficients in the video. Feel free to seek help if you have trouble.

\subsection*{Challenge}
Using the same approach as challenge \ref{sec:fcsinx}, deduce for the function $1+sin(x)$ the following Fourier coefficients:

\subsection*{Solutions}
$C_{-1}$: \hash{z}{39e026}

$C_0$: \hash{x}{0ef183}

$C_1$: \hash{c}{89b992}

\timebox




%%%%%%%%%%%%%%%%%%%%%%%%%%%%%%%%%
\newpage
%%%%%%%%%%%%%%%%%%%%%%%%%%%%%%%%%

\section{Relation of positive and negative Fourier coefficients for a real signal}

\subsection*{Resources}
\begin{itemize}
    \item Book: 1.4 (\url{https://see.stanford.edu/materials/lsoftaee261/book-fall-07.pdf})
    \item Video: Lecture 2 (\url{https://www.youtube.com/watch?v=1rqJl7Rs6ps})
\end{itemize}

\subsection*{Challenge}
If the Fourier coefficient $C_1$ is $4 + 6i$, what is the Fourier coefficient $C_{-1}$?

\subsection*{Solution}
X

\hash{m}{36ab38}

\timebox




\stepcounter{section}

%%%%%%%%%%%%%%%%%%%%%%%%%%%%%%%%%
\newpage
%%%%%%%%%%%%%%%%%%%%%%%%%%%%%%%%%

\section{The Fourier series of $f(t)=t$: $C_0$}

\subsection*{Resources}
\begin{itemize}
    \item Book: 1.5, 1.7 (\url{https://see.stanford.edu/materials/lsoftaee261/book-fall-07.pdf})
    \item Video: Lecture 3 (\url{https://www.youtube.com/watch?v=BjBb5IlrNsQ})
\end{itemize}

\subsection*{Challenge}
Considering the function $f(t)=t$ over the interval 0 to 1, calculate the Fourier coefficient $C_0$ using the derived formula for Fourier coefficients. Compare with the average over the interval.

\subsection*{Solution}
X

\hash{aa}{2708ad}

\timebox




%%%%%%%%%%%%%%%%%%%%%%%%%%%%%%%%%
\newpage
%%%%%%%%%%%%%%%%%%%%%%%%%%%%%%%%%

\section{The Fourier series of $f(t)=t$: $C_k$}

\subsection*{Resources}
\begin{itemize}
    \item Book: 1.5, 1.7 (\url{https://see.stanford.edu/materials/lsoftaee261/book-fall-07.pdf})
    \item Video: Lecture 3 (\url{https://www.youtube.com/watch?v=BjBb5IlrNsQ})
\end{itemize}

\subsection*{Challenge}
Considering the function $f(t)=t$, calculate a general expression for the Fourier coefficients $C_k$ where $k \ne 0$.

To check your answer, evaluate the Fourier coefficient for $k=-30$.

\subsection*{Solution}
X

\hash{bb}{8005c7}

\timebox




%%%%%%%%%%%%%%%%%%%%%%%%%%%%%%%%%
\newpage
%%%%%%%%%%%%%%%%%%%%%%%%%%%%%%%%%

\section{The Fourier series of $f(t)=t$ in exponential form}
\label{sec:fstexpform}

\subsection*{Resources}
\begin{itemize}
    \item Book: 1.4 (\url{https://see.stanford.edu/materials/lsoftaee261/book-fall-07.pdf})
    \item Video: Lecture 3 (\url{https://www.youtube.com/watch?v=BjBb5IlrNsQ})
\end{itemize}

\subsection*{Challenge}
1. Write $e^{i 2 \pi k}$ in terms of cosines and sines.

2. Evaluate your expression obtained in (1) for $k=0,1,2,3,4$

3. Write the function $f(t)=t$ evaluated between 0 and 1 in terms of its exponential Fourier series $f(t)=\sum_{k=-N}^{k=N} g(k,t)$ replacing $g(k,t)$ as appropriate.

% NT: Add a challenge with a sum from k=-n to n with n=2
To check your answer, evaluate the Fourier series up to $N=1$ with $t=0.8$.

The graph with increasing values of $N$ looks like this:

\includegraphics{fourier_series_t.png}

\subsection*{Solution}
X

\hash{cc}{caa033}

\timebox




%%%%%%%%%%%%%%%%%%%%%%%%%%%%%%%%%
\newpage
%%%%%%%%%%%%%%%%%%%%%%%%%%%%%%%%%

\section{The Fourier series of $f(t)=t$ in trigonometric form}
\label{sec:trigexpconvert}

\subsection*{Resources}
\begin{itemize}
    \item Book: 1.5, 1.7 (\url{https://see.stanford.edu/materials/lsoftaee261/book-fall-07.pdf})
    \item Video: Lecture 2 (\url{https://www.youtube.com/watch?v=1rqJl7Rs6ps})
\end{itemize}

\subsection*{Comment}
Since cosine and sine can be written in terms of exponentials, it is possible to switch between Fourier series that are expressed in terms of exponentials and Fourier series that are expressed in terms of sines and cosines. Many textbooks will actually work in terms of these trigonometric forms. Where only sine terms are involved, it is called a ``Fourier sine series'' and where only cosines are involved it's termed a ``Fourier cosine series''. The series have coefficients $a_0$, $a_k$ and $b_k$, in the following fashion:

\begin{equation}
    \label{eq:fsoftintrigform}
    f(t) = \frac{a_0}{2} + \sum_{k=1}^{k=N} a_k cos(2 \pi k t) + b_k sin(2 \pi k t)
\end{equation}

Note that the sum here goes from $k=1$ to $k=N$, in contrast to the exponential form of Fourier series which goes between $k=\pm N$.

\subsection*{Challenge}
1. Write $sin(x)$ in the form of an exponential sum:
\begin{equation}
    sin(x)=\sum_{k=-N}^{k=N} f(k) e^{g(k,x)}
\end{equation}
Your answer in challenge \ref{sec:fcsinx} may help you.

What is $f(k)$, $g(k,x)$ and $N$? To check your answers for $f(k)$ and $g(k,x)$, substitute $k=1$ and $x=2$ as appropriate into your expressions for $f(k)$ and $g(k,x)$.

2. By converting from an exponential-form sum ($\sum_{k=-N}^{k=N}$) to a trigonometric-form sum ($\sum_{k=1}^{k=N}$), re-write the series obtained in challenge \ref{sec:fstexpform} in terms of a trigonometric infinite series (ie, using sines and cosines). To check your answer, evaluate the Fourier series with $N=1$ with $t=0.8$ and ensure that you get the same answer as you did for challenge \ref{sec:fstexpform}.
% NT: Put re-arranging practise into another challenge and make it in more depth (cosines, and both directions). Otherwise students try to use formula instead. Need practise about switching from sums from -infty to sums from 1

3. You should find you are left with an expression only in terms of sine or cosine. Which is it, and how is this related to even/odd functions?

\subsection*{Solution}
\emph{(See the hash examples about entering imaginary numbers)}

$f(k)$: \hash{iiif}{3979fa}

$g(k)$: \hash{iiig}{6cd239}

$N$: \hash{iiin}{e3f634}

\timebox




%%%%%%%%%%%%%%%%%%%%%%%%%%%%%%%%%
\newpage
%%%%%%%%%%%%%%%%%%%%%%%%%%%%%%%%%
\section{Periods other than unity}
\label{sec:nonunitperiods}

\subsection*{Resources}
\begin{itemize}
    \item Book: 1.6.1 (\url{https://see.stanford.edu/materials/lsoftaee261/book-fall-07.pdf})
\end{itemize}

\subsection*{Comment}
Please read the resource. There, $c_n$ is illustrated for an arbitrary period $T$ going from $0$ to $T$, but you do not need to start from $0$. If instead you want to approximate a function from time $T_0$ to $T_0+T$, you can simply swap the integration limits (ie, $T_0$ instead of $0$ and $T_0+T$ for $T$).

\subsection*{Challenge}
\subsubsection*{Part I}
1. By expanding the exponential out in terms of sine and cosine, determine the numerical value of $e^{i \pi k}$ for $k=0,1,2,3,4$.

2. Assuming $k$ can only be an integer, determine the value of $N$ in the formula $e^{i \pi k} = N^k$ and the value of $M$ in the formula $e^{i 2 \pi k} = M^k$.

\subsubsection*{Part II}
Determine $C_0$ and $C_k$ for the following square-wave functions using exponential fourier-series representation:

\vspace{2em}
1.
\begin{equation}
    f(t)=
    \begin{cases}
        1 & \text{for } 0<t<1 \\
        0 & \text{for } 1<t<2
    \end{cases}
\end{equation}

\includegraphics[scale=0.5]{fourier_series_square_wave_01_2.png}

\vspace{2em}
2.
\begin{equation}
    f(t)=
    \begin{cases}
        1 & \text{for } 0<t<\frac{1}{2} \\
        0 & \text{for } \frac{1}{2}<t<1
    \end{cases}
\end{equation}

\includegraphics[scale=0.5]{fourier_series_square_wave_00p5_1.png}

\vspace{2em}
3. \emph{(Do not try to simplify the exponentials beyond cosines and sines)}
\begin{equation}
    f(t)=
    \begin{cases}
        1 & \text{for } \frac{1}{8}<t<\frac{7}{8} \\
        0 & \text{for } \frac{7}{8}<t<\frac{9}{8}
    \end{cases}
\end{equation}

\includegraphics[scale=0.5]{fourier_series_square_wave_eighths.png}

\subsubsection*{Part III}
1. Express each of the square-waves above in terms of an exponential fourier series $f(t)=\sum_{k=-N}^{k=N} g(k,t)$, changing $g(k,t)$ for the exponential fourier series expression. To check your expressions, evaluate the series for $t=1$ with $N=1$.

\subsection*{Solution}
\subsubsection*{Part I}
1.\\
$k=0$: \hash{ddk0}{1b3914}\\
$k=1$: \hash{ddk1}{da2480}\\
$k=2$: \hash{ddk2}{67efb2}\\
$k=3$: \hash{ddk3}{26a591}\\
$k=4$: \hash{ddk4}{f09b1c}

2.\\
N: \hash{ddn}{f3550e}\\
M: \hash{ddm}{e84cb4}

\subsubsection*{Part II}
1.\\
$C_0$: \hash{dd1c0}{aab052}\\
$C_1$: $\displaystyle \frac{-i}{\pi}$

2.\\
$C_0$: \hash{dd2c0}{c34c5d}\\
$C_1$: $\displaystyle \frac{-i}{\pi}$

3.\\
$C_0$: \hash{dd3c0}{ac102c}\\
$C_1$: $\displaystyle \frac{-1}{\sqrt{2}\pi}$

\subsubsection*{Part III}
1. First square-wave: \hash{dd1sum}{35788a}\\
2. Second square-wave: \hash{dd2sum}{75b60f}\\
3. Third square-wave: 0.30 (in decimal form, but it can be written more neatly with fractions and roots)

\timebox




%%%%%%%%%%%%%%%%%%%%%%%%%%%%%%%%%
\newpage
%%%%%%%%%%%%%%%%%%%%%%%%%%%%%%%%%
\section{Infinite series}

\subsection*{Resources}
\begin{itemize}
    \item Book: 1.7 (\url{https://see.stanford.edu/materials/lsoftaee261/book-fall-07.pdf})
\end{itemize}

\subsection*{Comment}
It is important to understand why some series are infinite, while others are not (well, technically all series are infinite since they all involve sums to $n=\infty$, however for some series the Fourier coefficients are all zero above a certain value of $n$). Here

\subsection*{Challenge}
1. In challenge \ref{sec:fcsinx} and \ref{sec:fcsinxp1} you determined the fourier coefficients for $sin(x)$ and $sin(x)+1$. If you write the function in the form
\begin{equation}
    \label{eq:sinxform}
    sin(x)=\sum_{k=-N}^{k=N} C_k e^{i k x}
\end{equation}
what is $N$ here?

\vspace{1em}
2. Expand $e^{i k x}$ in terms of sine and cosine. Which has the higher frequency? $k=1$ or $k=100$?

\vspace{1em}
3. Referring to the resource, do sharp corners in a function lead to higher or lower frequencies?

\vspace{1em}
4. Does non-periodiciy lead to higher or lower frequencies? In challenge \ref{sec:fstexpform} you calculated the Fourier series for $f(t)=t$. If the series is written in a form similar to equation \ref{eq:sinxform}, what would $N$ be in this case?

\vspace{1em}
5. In general, the Fourier series is a sum to $\pm \infty$, however in some cases the coefficients ($C_k$'s) are zero beyond a certain number of terms. Which of the functions below will have Fourier coefficients that are all zero after a certain number of terms? Sum the points of these functions.

1 point: $x$

2 points: $x^2$

4 points: $cos(2 x) + 3 sin(7 x)$

8 points: $e^{2 \pi i x}$

\vspace{1em}
6. Briefly explain the characteristics of functions that lead to infinite Fourier series and finite Fourier series.

\subsection*{Solution}
1. (enter as an integer, without ``.00'') \hash{ee1}{b6cadd} 

2. (``1'' or ``100'') \hash{ee2}{a0bebe}

3. (``lower'' or ``higher'') \hash{ee3}{f0d1f9}

4. \hash{ee4}{7cd9d2}

5. \hash{ee5}{9d1559}

\timebox




%%%%%%%%%%%%%%%%%%%%%%%%%%%%%%%%%
\newpage
%%%%%%%%%%%%%%%%%%%%%%%%%%%%%%%%%
\section{k-symmetry}

\subsection*{Challenge}
Determine what X and Y represent algebraically.
\begin{equation}
    cos(k \pi t) = \frac{1}{2} e^{-k i\pi t} + \frac{1}{2} e^{\bm{X} i \pi x}
\end{equation}
\begin{equation}
    sin(k \pi t) = \frac{1}{2} i e^{\bm{Y} i \pi t} - \frac{1}{2} i e^{k i \pi t}
\end{equation}

To check your answers you may substitute any appropriate values from the following list: $k=2$, $t=1$

\subsection*{Solution}
X

\hash{ff}{942d6f}

Y

MD5(gg\_Y) = a379b8\ldots

\timebox




%%%%%%%%%%%%%%%%%%%%%%%%%%%%%%%%%
\newpage
%%%%%%%%%%%%%%%%%%%%%%%%%%%%%%%%%
\section{Direct trigonometric calculation of a Fourier series: the coefficients}
\label{sec:fs_squarewave}
% NT: Need practise about integration over intervals

\section*{Comment}
This challenge introduces several key concepts at once, including decoupling of integral intervals and periodicity, the concept of a square wave and direct trigonometric evaluation of Fourier series. If you can master this you'll be in a really strong position.

It is hopefully clear now that for real signals, due to the symmetry of the positive and negative k's, one can fully compose Fourier series in terms of sine and cosine. In challenge \ref{sec:trigexpconvert} we saw the formula for the function in terms of Fourier coefficients $a_0$, $a_n$ and $b_n$. While we will not use this approach, it is important to be able to utilise such a formulation since this is the way some books present it and some people have learnt it. Therefore, without proof, the coefficients can be calculated using

\begin{equation}
    a_k = \frac{2}{T} \int_{t_0}^{t_0+T} f(t) Cos(2 \pi k t/T)
\end{equation}
\begin{equation}
    b_k = \frac{2}{T} \int_{t_0}^{t_0+T} f(t) Sin(2 \pi k t/T)
\end{equation}

\subsection*{Challenge}
Using the direct trigonometric Fourier series, obtain a general expression for the $a_k$ and $b_k$ coefficients for the square-wave signal with periodicity 4:

\begin{equation}
    f(t)=
    \begin{cases}
        1 & \text{for } -1<t<1 \\
        0 & \text{for } 1<t<3
    \end{cases}
\end{equation}

Note the symmetry of the problem. Can you see what terms will be zero? To check your solution, calculate $a_k$ and $b_k$ for $k=0$, $k=2$ and $k=3$. Note that you will have to break the integrals into two parts and sum them in order to tackle this problem.

A graph of the function, including the solution for various values of $n$, is shown here:

\includegraphics[scale=0.75]{fs_square_wave.png}

\subsection*{Solution}
\begin{tabular}{|l|l|l|}
    \hline
    $k$ & $a_k$ & $b_k$ \\
    \hline
    0 & MD5(hh\_X)=e57c15\ldots & MD5(ii\_X)=377fe2\ldots \\
    2 & MD5(jj\_X)=54aaa1\ldots & MD5(kk\_X)=be063f\ldots \\
    3 & MD5(mm\_X)=b8fce7\ldots & MD5(nn\_X)=b6fbaf\ldots \\
    \hline
\end{tabular}

\timebox




%%%%%%%%%%%%%%%%%%%%%%%%%%%%%%%%%
\newpage
%%%%%%%%%%%%%%%%%%%%%%%%%%%%%%%%%
\section{Direct trigonometric calculation of a Fourier series: the series}

\subsection*{Comment}
For a series with non-unit period, the Fourier series in trigonometric form given in equation \ref{eq:fsoftintrigform} can be modified to read

\begin{equation}
    \label{eq:fstrignonunit}
    f(t) = \frac{a_0}{2} + \sum_{k=1}^{k=N} a_k cos(2 \pi k t/T) + b_k sin(2 \pi k t/T)
\end{equation}

\subsection*{Challenge}
Calculate the Fourier series for the square wave introduced in challenge \ref{sec:fs_squarewave} using direct trigonometric calculation for up to $k=3$. Check your solution by evaluating for $t=0.1$.

\subsection*{Solution}
0.9397

%\hash{oo}{bd7e5e}

\timebox




%%%%%%%%%%%%%%%%%%%%%%%%%%%%%%%%%
\newpage
%%%%%%%%%%%%%%%%%%%%%%%%%%%%%%%%%
\section{2D orthogonal vectors}

\subsection*{Resources}
\begin{itemize}
    \item Book: 1.9 (\url{https://see.stanford.edu/materials/lsoftaee261/book-fall-07.pdf})
\end{itemize}

\subsection*{Challenge}
Sum the points of the vectors in 2D that are orthogonal:

1 point: (5, 4) and (-1, 1.25)

2 points:  (2, -3) and (-6, 4)

4 points: (-2.25, 1.5) and (2, 3)

8 points: (4.5, 4) and (3, -3.375)

16 points: (6, 4) and (4, -6)

32 points: (5, 1) and (-2, 8.125)

64 points: (0, 1) and (1, 0)

128 points: (1, 1) and (1, 1)

\subsection*{Solution}
\six{}

\hash{pp}{92843f}

\timebox




%%%%%%%%%%%%%%%%%%%%%%%%%%%%%%%%%
\newpage
%%%%%%%%%%%%%%%%%%%%%%%%%%%%%%%%%
\section{Orthonormal basis}

\subsection*{Resources}
\begin{itemize}
    \item Video: \url{https://www.khanacademy.org/math/linear-algebra/alternate-bases/orthonormal-basis/v/linear-algebra-introduction-to-orthonormal-bases}
\end{itemize}

\subsection*{Challenge}
Sum the points of the following vectors that form an orthonormal basis:

1 point :
($\displaystyle \frac{1}{\sqrt{5}}, \frac{2}{\sqrt{5}}$) and
($\displaystyle \frac{2}{\sqrt{5}}, \frac{4}{\sqrt{5}}$)

2 points:
($\displaystyle \frac{2}{\sqrt{5}}$, $\displaystyle \frac{1}{\sqrt{5}}$) and
($\displaystyle \frac{-1}{\sqrt{5}}$, $\displaystyle \frac{2}{\sqrt{5}}$)

4 points:
($\displaystyle \frac{2}{\sqrt{2}}, \sqrt{\frac{7}{8}}, \frac{1}{\sqrt{6}}$),
($\displaystyle -\sqrt{\frac{2}{5}}, \frac{7}{\sqrt{14}}, -\frac{1}{\sqrt{6}}$) and
($\displaystyle \frac{1}{\sqrt{3}},  \frac{1}{5 \sqrt{3}}, -\frac{7}{5 \sqrt{3}}$)

8 points:
($\displaystyle \frac{1}{\sqrt{21}}, \frac{2}{\sqrt{21}}, \frac{4}{\sqrt{21}}$),
($\displaystyle -\sqrt{\frac{2}{7}}, \frac{3}{\sqrt{14}}, -\frac{1}{\sqrt{14}}$) and
($\displaystyle \sqrt{\frac{2}{3}},  \frac{1}{\sqrt{6}}, -\frac{1}{\sqrt{6}}$)

16 points:
($\displaystyle \frac{1}{\sqrt{6}}, \sqrt{\frac{2}{3}}, \frac{1}{\sqrt{6}}$),
($\displaystyle -\frac{1}{\sqrt{2}},  \frac{2 \sqrt{2}}{5}, -\frac{3}{5 \sqrt{2}}$) and
($\displaystyle \frac{1}{\sqrt{3}},  \frac{1}{5 \sqrt{3}}, -\frac{7}{5 \sqrt{3}}$)

32 points:
($0, 2$) and ($2, 0$)

64 points:
($0, 1$) and ($1, 0$)

\subsection*{Solution}
\six{}

\hash{qq}{097fd7}

\timebox




%%%%%%%%%%%%%%%%%%%%%%%%%%%%%%%%%
\newpage
%%%%%%%%%%%%%%%%%%%%%%%%%%%%%%%%%
\section{Natural basis}

\subsection*{Resources}
\begin{itemize}
    \item Book: 1.9 (\url{https://see.stanford.edu/materials/lsoftaee261/book-fall-07.pdf})
\end{itemize}

\subsection*{Challenge}
Sum the components of the following vectors of an orthonormal basis in $\mathbb{R}^{300}$ space:

\begin{itemize}
    \item First component of the first vector
    \item first component of the second vector
    \item 200th component of the 100th vector
    \item 200th component of the 200th vector
    \item last component of the 299th vector
    \item last component of the last vector
\end{itemize}

\subsection*{Solution}
\six{}

\hash{rr}{095c77}

\timebox




%%%%%%%%%%%%%%%%%%%%%%%%%%%%%%%%%
\newpage
%%%%%%%%%%%%%%%%%%%%%%%%%%%%%%%%%
\section{Orthonormal basis for Fourier series}

\subsection*{Resources}
\begin{itemize}
    \item Book: 1.9 (\url{https://see.stanford.edu/materials/lsoftaee261/book-fall-07.pdf})
    \item Video: Lecture 4 (\url{https://www.youtube.com/watch?v=n5lBM7nn2eA})
\end{itemize}

\subsection*{Comment}
The previous challenges have focussed on the orthogonality and orthonormality of vectors. We now make the jump to functions. As chapter 1.9 explains, although its not perfect, the analogy between vectors and functions is a good way to help understand and visualise the role that the terms of a Fourier series play in defining a basis upon which to describe a function.

\subsection*{Challenge}
Starting from the inner product of two terms ($e^{2 \pi i k_1 t}$, $e^{2 \pi i k_2 t}$) of a Fourier series, demonstrate that the terms of a Fourier series form an orthonormal basis. \textbf{Show a full derivation}.

To check your intuition, you may evaluate the following cases:

$X = (e^{2 \pi i k_1 t}, e^{2 \pi i k_1 t})$

$Y = (e^{2 \pi i k_1 t}, e^{2 \pi i k_2 t})$

\subsection*{Solution}
If you are not confident about your derivation, please check with someone else. If there is any step that you do not fully understand, do not hesitate to ask. If you do not understand the connection between previous challenges on vectors and this challenge using functions, do not hesitate to ask someone.

\textbf{X}

\hash{ss}{8f7f41}

\textbf{Y}

MD5(tt\_Y) = 2c669b\ldots

\timebox




%%%%%%%%%%%%%%%%%%%%%%%%%%%%%%%%%
\newpage
%%%%%%%%%%%%%%%%%%%%%%%%%%%%%%%%%
\section{Circles and Fourier series}

\subsection*{Resources}
\begin{itemize}
    \item Video 1: \url{https://www.youtube.com/watch?v=Y9pYHDSxc7g}
    \item Video 2: \url{https://www.youtube.com/watch?v=LznjC4Lo7lE}
\end{itemize}

\subsection*{Comment}
In the first lecture we saw how it was possible to approximate any function given enough circles. Here we link what you have learned back to that first lecture. I strongly recommend viewing the fun and informative videos listed here under Resources. In summary, by building a Fourier series you are representing a function using an orthonormal basis, where each component of the basis can be considered visually as a circle operating with individual radius and frequency on the real-imaginary plane. If, after completing this challenge, that last sentence makes sense to you, then you have achieved the first major goal of this course.

\includegraphics[scale=0.5]{circle.png}

\subsection*{Challenge}
Treating the x-axis as the real axis and the y-axis as the imaginary axis, arrange the equations below in the following order:

\begin{enumerate}
    \item A point moving round on a circle with radius 2 units and frequency 2 Hz
    \item A point moving round on a circle with radius 3 units and frequency 1 Hz
    \item A point moving round on a circle with radius 2 units and a period of 1 second
    \item A point moving round on a circle with radius 3 units and a period of 2 seconds
\end{enumerate}

Equations:

$\displaystyle A e^{2 \pi i k t}$ where $t$ is time in seconds and the values of $A$ and $k$ are as follows:

A: $A=2$, $k=2$ 

B: $A=3$, $k=1$

C: $A=3$, $k=0.5$

D: $A=2$, $k=1$

\subsection*{Solution}
\six{}

\hash{uu}{cb7845}

\timebox




%%%%%%%%%%%%%%%%%%%%%%%%%%%%%%%%%
\newpage
%%%%%%%%%%%%%%%%%%%%%%%%%%%%%%%%%
\section{Gibb's phenomenon}

\subsection*{Resources}
\begin{itemize}
    \item Wikipedia: \url{https://en.wikipedia.org/wiki/Gibbs_phenomenon}
    \item Book: 1.18 (\url{https://see.stanford.edu/materials/lsoftaee261/book-fall-07.pdf})
\end{itemize}

\subsection*{Challenge}
Write a few sentences summarising your understanding of what Gibb's phenomenon is. By what percentage does overshoot of a discontinuity of a square-wave function occur?

\emph{The full derivation is beyond the scope of this course, so it is not necessary to understand the (rather complex) derivation in the notes. The aim of this challenge is simply to enable you to be able to describe qualitatively what Gibb's phenomenon is using a few sentences, and know the amount of overshoot in the case of a standard square-wave. No maths expected.}

\subsection*{Solution}
\six{\%}

\hash{vv}{aa19f2}

\timebox



%%%%%%%%%%%%%%%%%%%%%%%%%%%%%%%%%
\newpage
%%%%%%%%%%%%%%%%%%%%%%%%%%%%%%%%%
\section{Partial derivatives}

\subsection*{Challenge}
Determine $u_t$ and $u_{xx}$ for the equation

\begin{equation}
    u(x,t) = 5tx^2 + 3t - x
\end{equation}

To check your answer, substitute $x=3$ and $t=2$ into your answers, as appropriate.

\subsection*{Solution}
$u_t$: \hash{ww}{7901cb}

$u_{xx}$: \hash{xx}{6aba1c}

\timebox




%%%%%%%%%%%%%%%%%%%%%%%%%%%%%%%%%
\newpage
%%%%%%%%%%%%%%%%%%%%%%%%%%%%%%%%%
\section{Heat equation: Periodicity}

\subsection*{Resources}
\begin{itemize}
    \item Lecture 4 from 37:00 onwards: (\url{https://www.youtube.com/watch?v=n5lBM7nn2eA}), continuing at the start of lecture 5 (\url{https://www.youtube.com/watch?v=X5qRpgfQld4})
\end{itemize}

\subsection*{Comment}
Please follow the derivation of the heat equation shown in the videos in lectures 4-5. This use of Fourier series is a great example of how the relatively abstract mathematical concepts covered by the course so-far can have real physical applications in science and engineering.

\subsection*{Challenge}
Add the points of the following true statements concerning a heated ring with circumference 1 and temperature distribution described by $u(x,t)$:

1 point: $\displaystyle u(x,t) = u(x,t)$

2 points: $\displaystyle u(x,t) = u(x,t+1)$

4 points: $\displaystyle u(x,t) = u(x,t+2)$

8 points: $\displaystyle u(x,t) = u(x+1,t)$

16 points: $\displaystyle u(x,t) = u(x+1,t+1)$

32 points: $\displaystyle u(x,t) = u(x+1,t+2)$

64 points: $\displaystyle u(x,t) = u(x+2,t)$

128 points: $\displaystyle u(x,t) = u(x+2,t+1)$

256 points: $\displaystyle u(x,t) = u(x+2,t+2)$

512 points: The temperature distribution is periodic in space but not time

1024 points: The temperature distribution is periodic in time but not space

2048 points: The temperature distribution is periodic in both space and time

4096 points: The temperature distribution is periodic neither in space nor time


\subsection*{Solution}
\six{}

\hash{yy}{ed4019}

\timebox




%%%%%%%%%%%%%%%%%%%%%%%%%%%%%%%%%
\newpage
%%%%%%%%%%%%%%%%%%%%%%%%%%%%%%%%%
\section{Heat equation: Fourier coefficients}
\label{sec:heateqnfc}

\subsection*{Resources}
\begin{itemize}
    \item Lecture 4 from 37:00 onwards: (\url{https://www.youtube.com/watch?v=n5lBM7nn2eA}), continuing at the start of lecture 5 (\url{https://www.youtube.com/watch?v=X5qRpgfQld4})
\end{itemize}

\subsection*{Challenge}
Starting from the heat (diffusion) equation $u_t = u_{xx}/2$, show that the Fourier coefficient at time $t$ is given by

\begin{equation}
    C_k(t) = C_k(0) e^{-2 \pi^2 k^2 t}
\end{equation}


\subsection*{Solution}
Compare with your peers during discussion time and please ask if there is anything you do not understand.

\timebox




%%%%%%%%%%%%%%%%%%%%%%%%%%%%%%%%%
\newpage
%%%%%%%%%%%%%%%%%%%%%%%%%%%%%%%%%
\section{Heat equation}
\label{sec:heateqn}

\subsection*{Resources}
\begin{itemize}
    \item Lecture 4 from 37:00 onwards: (\url{https://www.youtube.com/watch?v=n5lBM7nn2eA}), continuing at the start of lecture 5 (\url{https://www.youtube.com/watch?v=X5qRpgfQld4})
\end{itemize}

\subsection*{Comment}
Note that the exponential in the answer for $B(t)$ is negative, leading to a decay to mean ambient temperature over time (you can think of the temperature $u(x,t)$ as the temperature relative to the surrounding environment rather than absolute temperature measured in Kelvin).

\subsection*{Challenge}
Write the heat equation and its Fourier coefficient in the form below. Identify A(x), B(t) and C(x). To check your answers, substitute $k=1$, $x=2$ and $t=3$ into the expressions (purely for checking the solution; these numbers have no physical basis).
% This is not ideal because it doesn't distinguish negative signs

\begin{equation}
    \hat{f}(k) = \int_0^1 A(x) f(x) dx
\end{equation}

\begin{equation}
    u(x,t) = \sum_{k=-\infty}^{k=\infty} \hat{f}(k) B(t) C(x)
\end{equation}

\subsection*{Solution}
A: \hash{zz}{ffef92}

B: \hash{aaa}{c04067}

C: \hash{bbb}{ae8c65}

\timebox




%%%%%%%%%%%%%%%%%%%%%%%%%%%%%%%%%%
%\newpage
%%%%%%%%%%%%%%%%%%%%%%%%%%%%%%%%%%
%\section{Heat equation derivation}
%
%\subsection*{Resources}
%\begin{itemize}
%    \item Lecture 4 from 37:00 onwards: (\url{https://www.youtube.com/watch?v=n5lBM7nn2eA}), continuing at the start of lecture 5 (\url{https://www.youtube.com/watch?v=X5qRpgfQld4})
%\end{itemize}
%
%\subsection*{Challenge}
%Continuing from challenge \ref{sec:heateqnfc}, derive the expression found in challenge \ref{sec:heateqn} for the temperature on a ring of unit length as a function of position and time using Fourier series.
%
%\subsection*{Solution}
%Please check your derivation with your peers. The teacher will also review your derivation during challenge-log collection.
%
%\timebox




% So we have a fourier series in space (with FC's given by C_k when t=0) that decays over time
% Note f^(k) is the equivalent of c_k
% Calculate the heat around a ring



% Convergence?



% Sine and cosine series? Even odd functions?
%\chapter{Fourier Transform (continuous)}
%\chapter{Fourier Transform (discrete)}
% Discrete FT
% FFT
% Zero-padding

% Convolution?
\end{document}
