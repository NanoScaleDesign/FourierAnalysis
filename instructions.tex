\section{This course}
This is the Autumn 2016 \course course studied by \nensei-year undegraduate international students at Kyushu University.

\subsection{How this works}
\begin{itemize}
    \item In contrast to the traditional lecture-homework model, in this course the learning is self-directed and active via publicly-available resources.
    \item Learning is guided through solving a series of carefully-developed challenges contained in this book (download from \url{http://www.jamescannon.net/teaching/\courseurl}), coupled with suggested resources that can be used to solve the challenges with instant feedback about the correctness of your answer.
    \item There are no lectures. Instead, there is discussion time. Here, you are encouraged to discuss any issues with your peers, teacher and any teaching assistants. Furthermore, you are encouraged to help your peers who are having trouble understanding something that you have understood; by doing so you actually increase your own understanding too.
    \item Peer discussion is encouraged, however, if you have help to solve a challenge, always make sure you do understand the details yourself. You will need to be able to do this in an exam environment. If you need additional challenges to solidify your understanding, then ask the teacher. The questions on the exam will be similar in nature to the challenges. If you can do all of the challenges, you can get 100\% on the exam.
    \item Discussion-time is from \disctime on \discdays at room \discroom.
    \item Every challenge in the book typically contains a \textbf{Challenge} with suggested \textbf{Resources} which you are recommended to utilise in order to solve the challenge. A \textbf{Solution} is made available in encrypted form. If your encrypted solution matches the encrypted solution given, then you know you have the correct answer and can move on. For more information about encryption, see section \ref{sec:hashes}. Occasionally the teacher will provide extra \textbf{Comments} to help guide your thinking.
    \item For deep understanding, it is recommended to study the suggested resources beyond the minimum required to complete the challenge.
    \item The challenge document has many pages and is continuously being developed. Therefore it is advised to view the document on an electronic device rather than print it. The date on the front page denotes the version of the document. You will be notified by email when the document is updated.
    \item A “target challenge” and “minimum challenge” will be set each week, to be achieved by the beginning of the discussion time.
    \begin{itemize}
        \item Target challenge: You are expected to complete at least up to this challenge. This is because the “group learning” effect is strongest when everyone is roughly around the same level of understanding.
        \item Minimum challenge: Due to personal (eg health) or professional (eg conference attendance) issues, or simply difficulties with the challenges, it may occasionally not be possible to reach the target challenge. In this case, you will still be considered to be progressing normally if you achieve the minimum challenge.
        \item You may work ahead, even beyond the target challenge, if you so wish. This can build greater flexibility into your personal schedule, especially as you become busier towards the end of the semester.
    \end{itemize}
    \item Your contributions to the course are strongly welcomed. If you come across resources that you found useful that were not listed by the teacher or points of friction that made solving a challenge difficult (there's no such thing as ``you should have learned it in high-school'' - you're probably not the only one with that specific problem), please let the teacher know about it!
\end{itemize}

\subsection{Assessment}
In order to prove to outside parties that you have learned something from the course, we must perform summative assessments. This will be in the form of
\begin{itemize}
    \item Exam(s): Final exam, and possibly a mid-term exam
    \item Group presentation towards the end of the course (details to be announced later)
\end{itemize}

\subsection{What you need to do}
\begin{itemize}
    \item Prepare a challenge-log in the form of a workbook or folder where you can clearly write the calculations you perform to solve each challenge. This will be a log of your progress during the course and will be occasionally reviewed by the teacher.
    \item You will need to maintain a google spreadsheet detailing your work and progress. The purpose of this spreadsheet is to help the teacher optimise the discussion-time. Please ensure that it is up-to-date 24 hours before each discussion-time starts. It is fine for you to continue to work on challenges and update the spreadsheet after the 24-hour deadline.
    \item You also need to submit a brief report at \url{https://goo.gl/forms/Djl4FEZcJLMpipsY2} 24 hours before the discussion time starts. Here you can let the teacher know about any difficulties you are having and if you would like to discuss anything in particular.
    \item Please bring a wifi-capable internet device to class, as well as headphones if you need to access online components of the course during class. If you let me know in advance, I can lend computers and provide power extension cables for those who require them (limited number).
\end{itemize}

\subsection{Details about the spreadsheet}
To get started:
\begin{enumerate}
    \item Log into google
    \item Open \url{http://bit.ly/2cPYyQY}
    \item File $\rightarrow$ Make a copy [$\rightarrow$ rename] $\rightarrow$ ok
    \item Click ``Share'' (top right)
    \item Click ``Get shareable link''
    \item Set ``Anyone with the link can edit''
    \item Copy sharing address
    \item Send an email to cannon@mech.kyushu-u.ac.jp containing
    \begin{enumerate}
       \item Subject: \course registration
       \item Your name
       \item Student number
       \item The link to your copy of the google sheet
    \end{enumerate}
\end{enumerate}

Using the spreadsheet:

\begin{itemize}
    \item Enter the appropriate challenge number. For example, for challenge 1.4, enter ``1'' in the \textbf{Section} column and ``4'' in the \textbf{Challenge} column.
    \item After successfully completing a challenge, please enter any particular friction points that you experienced (if any) so the course can be developed to reduce such friction in the future, as well as any extra resources you recommend (if any).
    \item Please also roughly estimate the amount of effort in \textbf{Hours} required to complete the challenge (starting from when you completed the previous challenge, including any reading, watching videos, looking for resources, writing the answer to the challenge, discussing with peers, etc). This is not used for assessment in any way, but is very valuable in helping the teacher develop the course.
\end{itemize}

Note: Please do not alter column names, ordering, etc. Just add section and challenge numbering and fill in the columns as appropriate. This is because spreadsheet data is downloaded and automatically analysed, and it breaks if anything is inconsistent.
