\section{Hash practise: Integer}

X = 46.3847\\
Form: Integer.\\
Place the indicated letter in front of the number.\\
Example: aX where $X=46$ is entered as \href{http://www.wolframalpha.com/input/?i=md5+hash+of+\%22a46\%22}{a46}

hash of aX = e77fac

\section{Hash practise: Decimal}

X = 49\\
Form: Two decimal places.\\
Place the indicated letter in front of the number.\\
Example: aX where $X=46.00$ is entered as \href{http://www.wolframalpha.com/input/?i=md5+hash+of+\%22a46\%22}{a46.00}

hash of bX = 82c9e7

\section{Hash practise: String}

X = abcdef\\
Form: String.\\
Place the indicated letter in front of the number.\\
Example: aX where $X=abc$ is entered as \href{http://www.wolframalpha.com/input/?i=md5+hash+of+\%22aabc\%22}{aabc}

hash of cX = 990ba0

\section{Hash practise: Scientific form}

X = 500,765.99\\
Form: Scientific notation with the mantissa in standard form to 2 decimal place and the exponent in integer form.\\
Place the indicated letter in front of the number.\\
Example: aX where $X=4 \times 10^{-3}$ is entered as \href{http://www.wolframalpha.com/input/?i=md5+hash+of+\%22a4.00e-3\%22}{a4.00e-3}

hash of dX = be8a0d

\section{Hash practise: Numbers with real and imaginary parts}

X = $1+2i$\\
Form: Integer \emph{(for imaginary numbers, ``integer'' means to write both the real and imaginary parts of the number as integers. If you were instructed to enter to two decimal places, then you would need to enter each of the real and imaginary parts to two decimal places. Refer to section \ref{sec:hashes} to see an example of how to handle input of imaginary numbers.)}\\
Place the indicated letter in front of the number.\\
Example: aX where $X=46$ is entered as \href{http://www.wolframalpha.com/input/?i=md5+hash+of+\%22a46\%22}{a46}

hash of eX = 4aa75a

\section{$\pi$ and imaginary numbers}

X = $-2\pi i$\\
Form: Two decimal places.
Place the indicated letter in front of the number.\\
Example: aX where $X=46.00$ is entered as \href{http://www.wolframalpha.com/input/?i=md5+hash+of+\%22a46\%22}{a46.00}

hash of fX = ad3e8b

\section{Imaginary exponentials}

Note that you will need to understand how to expand exponentials in terms of their sines and cosines in order to do this. If you do not understand how to do this yet, skip this challenge and come back to it later.

X = $4e^{i 3 \pi/4}$\\
Form: Two decimal places.
Place the indicated letter in front of the number.\\
Example: aX where $X=46.00$ is entered as \href{http://www.wolframalpha.com/input/?i=md5+hash+of+\%22a46\%22}{a46.00}

hash of gX = 59a753
